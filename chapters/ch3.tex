
\chapter{Signal Interface}
\clearpage
\section{PIPE Interface}

\label{sec:1}
\begin{table}[H]
    \caption{TX and Related signals}
    \label{tab:p1}
    \centering
  \begin{tabular}{ |m{26mm}|m{10mm}|m{60mm}|  }
\hline
\rowcolor{Gray}
\multicolumn{1}{|c|}{\textbf{Name} } 
& \multicolumn{1}{|c|}{\textbf{Active Level}} 
& \multicolumn{1}{|c|}{\textbf{Description}}\\
\hline
TxData[7:0] 
& 
N/A
&
Parallel data input bus. \\
\hline

TxDataValid
& 
N/A
&
This signal allows the MAC to instruct the PHY to ignore the data interface \newline
for one clock cycle. A value of one \newline
indicates the phy will use the data, a \newline
value of zero indicates the phy will not \newline
use the data. \\
\hline

TxElecIdle
& 
High
&
Forces Tx output to electrical idle when asserted
except in loopback.\newline \newline
Note: The MAC must always have TxDataValid
asserted when TxElecIdle transitions to either
asserted or deasserted; TxDataValid is a qualifier
for TxElecIdle sampling.
\\
\hline


TxDataK
& 
N/A
&
A value of zero indicates a
data byte, a value of 1 indicates a
control byte.\\
\hline


TxStartBlock
& 
N/A
&
This signals allow the MAC to tell
the PHY the starting byte for a 128b
block when value is 1. The starting byte for a 128b
block must always start with byte 0 of
the data interface. \\
\hline



TxSyncHeader[1:0]
& 
N/A
&

Provides the sync header for
the PHY to use in the next 130b block.
\newline \newline
$ 01 \longrightarrow $control byte \newline
$ 10 \longrightarrow $data byte

\\
\hline


TxDetectRx/ \newline Loopback
& 
High
&
Used to tell the PHY to begin a receiver detection
operation or to begin loopback \\
\hline



\end{tabular}

\end{table}


% Rx and related signal
\begin{table}[H]
    \caption{RX and Related signals}
    \label{tab:p2}
    \centering
  \begin{tabular}{ |m{26mm}|m{10mm}|m{60mm}|  }
\hline
\rowcolor{Gray}
\multicolumn{1}{|c|}{\textbf{Name} } 
& \multicolumn{1}{|c|}{\textbf{Active Level}} 
& \multicolumn{1}{|c|}{\textbf{Description}}\\
\hline
RxData[7:0] 
& 
N/A
&
Parallel data output bus. \\
\hline

RxDataValid
& 
N/A
&
This signal allows the PHY to instruct the MAC to ignore the data interface \newline
for one clock cycle. A value of one \newline
indicates the mac will use the data, a \newline
value of zero indicates the mac will not \newline
use the data. \\
\hline

RxElecIdle
& 
High
&
Indicates receiver detection of an
electrical idle. While deasserted
with the PHY in P2 (PCI Express
mode).
\\
\hline


RxDataK
& 
N/A
&
A value of zero indicates a
data byte, a value of 1 indicates a
control byte.\\
\hline


RxStartBlock
& 
N/A
&
This signals allow the PHY to tell
the MAC the starting byte for a 128b
block when value is 1. The starting byte for a 128b
block must always start with byte 0 of
the data interface. \\
\hline



RxSyncHeader[1:0]
& 
N/A
&

Provides the sync header for
the MAC to use in the next 130b block.
\newline \newline
$ 01 \longrightarrow $control byte \newline
$ 10 \longrightarrow $data byte

\\
\hline


RxValid
& 
High
&
Indicates symbol lock and valid
data on RxData and RxDataK and
further qualifies RxDataValid
when used.\\
\hline


RxStatus[2:0]
& 
N/A
&
Encodes receiver status and error
codes for the received data
stream when receiving data.\newline
\begin{tabular}{|m{2mm}|m{2mm}|m{2mm}|m{30mm}| }
 \hline
[0]&[1]&[2]& Description \\
     \hline
0 & 0 & 0 & Received data OK \\
     \hline
0 & 0 & 1 & 1 SKP added \\
     \hline
0 & 1 & 0 & 1 SKP removed \\
     \hline
0 & 1 & 1 & Receiver detected\\
     \hline
1 & 0 & 0 & Both $8B/10B$  $(128B/130B 6 )$ decode 
error and (optionally) 
Receive Disparity error \\
     \hline
1 & 0 & 1 & Elastic Buffer overflow \\
     \hline

1 & 1 & 0 & Elastic Buffer
underflow. \\
     \hline
1 & 1 & 1 & Receive disparity error
(Reserved if Receive
Disparity error is
reported with code
0b100)\\
     \hline

\end{tabular}

\\
\hline


\end{tabular}


\end{table}
\begin{table}[H]

    \centering
  \begin{tabular}{ |m{26mm}|P{18mm}|m{60mm}|  }
  \hline

RxStandby & Low & Controls whether the PHY RX is active when the
PHY is in P0 or P0s. \newline \newline
$0 \longrightarrow Active$ \newline
$1 \longrightarrow Standby$
\newline
 \\
     \hline
 RxStandbyStatus & Low &  The PHY uses this signal to indicate its
RxStandby state.
 \newline \newline
$0 \longrightarrow Active$ \newline
$1 \longrightarrow Standby$
\newline
\\
     \hline

\end{tabular}
\end{table}




\begin{table}[H]
    \caption{Clk and reset}
    \centering
  \begin{tabular}{ |m{26mm}|m{10mm}|m{60mm}|  }
\hline
\rowcolor{Gray}
\multicolumn{1}{|c|}{\textbf{Name} } 
& \multicolumn{1}{|c|}{\textbf{Active Level}} 
& \multicolumn{1}{|c|}{\textbf{Description}}\\
\hline
 Reset\# & Low & Resets the transmitter and receiver. This signal
is asynchronous. \\
\hline

 CLK & Edge & This differential Input is used to generate
the bit-rate clock for the PHY transmitter
and receiver.\\
\hline
 PCLK & Rising Edge & All data movement across the parallel
interface is synchronized to this clock.\\
\hline

\end{tabular}
\end{table}


\begin{table}[H]
    \caption{Commands and Status signals}
    \centering
  \begin{tabular}{ |m{26mm}|m{10mm}|m{60mm}|  }
\hline
\rowcolor{Gray}
\multicolumn{1}{|c|}{\textbf{Name} } 
& \multicolumn{1}{|c|}{\textbf{Active Level}} 
& \multicolumn{1}{|c|}{\textbf{Description}}\\
\hline
 
PowerDown[3:0] M2P  & N/A & Power up or down the transceiver. Power states
PCI Express Mode: \newline
\begin{tabular}{|m{2mm}|m{2mm}|m{2mm}|m{2mm}|m{30mm}|}
 \hline
[0]&[1]&[2]&[3]& Description \\
     \hline
     
0& 0 & 0 & 0 & P0, normal operation \\
     \hline
0& 0 & 0 & 1 & P0s, low recovery time
latency, power saving state \\
     \hline

0& 0 & 1 & 0 & P2, lowest power state\\
     \hline

\end{tabular}

\\
\hline

 
Rate[3:0] \newline M2P
 & N/A & 
 Control the link signaling rate.
 
 \begin{tabular}{|c|c|}
    \hline
     Value & Description  \\ \hline
     0  & Use 2.5 GT/s signaling rate \\ \hline
     1 & Use 5 GT/s signaling rate \\ \hline
     2 & Use 8 GT/s signaling rate\\ \hline
     3 & Use 16 GT/s signaling rate\\ \hline
     4 & Use 32 GT/s signaling rate\\ \hline
 \end{tabular}
\\
\hline
 PHY Mode[3:0] \newline M2P & N/A & PHYMode[1:0]
 \newline \newline
 $0 \longrightarrow PCI Express $
 \newline
 \\
\hline

\end{tabular}
\end{table}

\begin{table}[H]

    \centering
  \begin{tabular}{ |m{26mm}|P{18mm}|m{60mm}|  }
  
\hline
 PhyStatus \newline P2M & High & Used to communicate completion
of several PHY functions including
stable PCLK and/or Max PCLK
(depending on clocking mode)
after Reset\# deassertion, power
management state transitions,
rate change, and receiver
detection.\\
\hline

Width[1:0] & N/A &Controls the PIPE data path width. \newline 
\begin{tabular}{|c|c|}
\hline
    Value  & Datapath Width  \\ \hline
    0 &  8 bits \\ \hline
    1 & 16 bits \\ \hline
    2 & 32 bits \\ \hline
\end{tabular}
\\
\hline
PCLK Rate[4:0] & N/A & Control the PIPE PCLK rate. \newline
\begin{tabular}{|c|c|}
\hline
    Value  & clk rate  \\ \hline
    0 & 32.25 Mhz \\ \hline
    1 &  62.5 Mhz \\ \hline
    2 & 125 Mhz \\ \hline
    3 & 250 Mhz \\ \hline
    4 & 500 Mhz \\ \hline
    5 & 1000 Mhz \\ \hline
    6 & 2000 Mhz \\ \hline
    7 & 4000 Mhz \\ \hline
    
    
\end{tabular}
\\
\hline

PclkChangeAck \newline M2P & High & 
Only used when PCLK is a PHY input.
Asserted by the MAC when a PCLK rate
change or rate change or, if required,
width change is complete and stable. \newline 

After the MAC asserts PclkChangeAck
the PHY responds by asserting
PhyStatus for one cyle and de-asserts
PclkChangeOk at the same time as
PhyStatus. The controller shall deassert
PclkChangeAck when PclkChangeOk is
sampled low.

\\
\hline
PclkChangeOk \newline P2M & High &
Only used when PCLK is a PHY
input. Asserted by the PHY when
it is ready for the MAC to change
the PCLK rate or Rate.
\\
\hline
\end{tabular}
\end{table}



\begin{table}[H]
    \caption{Message Bus Interface Signals}
    \centering
  \begin{tabular}{ |P{30mm}|P{10mm}|m{60mm}|  }
\hline
\rowcolor{Gray}
\multicolumn{1}{|c|}{\textbf{Name} } 
& \multicolumn{1}{|c|}{\textbf{Direction}} 
& \multicolumn{1}{|c|}{\textbf{Description}}\\
\hline

M2P\_MessageBus\newline[7:0] & Input & The MAC multiplexes command, any
required address, and any required data
for sending read and write requests to
access PHY PIPE registers and for
sending read completion responses and
write ack responses to PHY initiated
requests.  \\ \hline 
P2M\_MessageBus\newline[7:0] & Output & The PHY multiplexes command, any
required address, and any required data
for sending read and write requests to
access MAC PIPE registers and for
sending read completion responses and
write ack responses to MAC initiated
requests.  \\ \hline 

\end{tabular}
\end{table}



\section{LPIF Inteface}
\label{sec:2}
\begin{table}[H]
\label{tab:l1}
    \caption{LPIF Interface Signals}
    \centering
  \begin{tabular}{ |P{26mm}|P{10mm} | m{60mm}|  }
\hline
\rowcolor{Gray}
\multicolumn{1}{|c|}{\textbf{Name} } 

& \multicolumn{1}{|c|}{\textbf{Active Level}}
& \multicolumn{1}{|c|}{\textbf{Description}}
\\
\hline
lclk & High &Link Clock: The clock frequency the LPIF interface operates at.
The Link Clock is an input to signal to both the Link Layer as well as the Logical PHY.\\ \hline

pl\_trdy & High & LIndicates Physical Layer is ready to accept data. Data is accepted when pl\_trdy, lp\_valid, and lp\_irdy are asserted together.\\ \hline


lp\_irdy & High & Link Layer to Physical Layer indicating Link Layer is ready to transfer data. lp\_irdy must not be presented by the upper layers when pl\_state\_sts is RESET.\\ \hline


lp\_valid [LP\_NVLD-1:0] & High & Link Layer to Physical Layer indicates data valid on the corresponding lp\_data bytes.
‘LP\_NVLD’ equals the number of valid bits. The bytes of lp\_data associated with a
specific bit of lp\_valid is implementation specific. When lp\_irdy is asserted, at least one
of the bits of lp\_valid must be asserted.\\ \hline
pl\_data [NBYTES-1:0][7:0] & N/A &
Physical Layer to Link Layer Data, where ‘NBYTES’ equals number of bytes determined
by the supported data bus for the LPIF interface. \\ \hline
pl\_valid [PL\_NVLD-1:0] & High &
Physical Layer to Link Layer indicates data valid on pl\_data. ‘PL\_NVLD’ equals the
number of valid bits. The bytes of pl\_data associated with a specific bit of pl\_valid is
implementation specific.\\ \hline
pl\_stallreq & High & Physical Layer request to Link Layer to flush all packets for state transition \\ \hline
lp\_data [NBYTES-1:0][7:0] & N/A & Link Layer to Physical Layer Data, where ‘NBYTES’ equals number of bytes determined
by the data width for the LPIF instance.
\\ \hline

\end{tabular}
\end{table}

\begin{table}[H]

    \centering
  \begin{tabular}{ |m{26mm}|P{18mm}|m{60mm}|  }
  \hline

\hline
lp\_stallack & High & Link Layer to Physical layer indicates that the packets are aligned (if pl\_stallreq was
asserted) and logPHY may begin state transitions. \\ \hline 
lp\_state\_req[3:0] & N/A & 
Link Layer Request to Logical Physical Layer to request state change.
Encodings as follows: \newline
0000: NOP \newline 
0001: Active \newline 
0010: Active.L0s \newline 
0011: Deepest Allowable PM State [L1 Substates only] \newline 
0100: L1.1 \newline 
0101: L1.2 \newline 
0110: L1.3 \newline 
0111: L1.4 \newline 
1000: L2 \newline 
1001: LinkReset \newline 
1010: Reserved \newline 
1011: Retrain \newline 
1100: Disable \newline 
\\ \hline
pl\_state\_sts[3:0] & N/A &
Physical Layer to Link Layer Status indication of the Interface.
Encodings as follows:
\newline
0000: NOP \newline 
0001: Active \newline 
0010: Active.L0s \newline 
0011: Deepest Allowable PM State [L1 Substates only] \newline 
0100: L1.1 \newline 
0101: L1.2 \newline 
0110: L1.3 \newline 
0111: L1.4 \newline 
1000: L2 \newline 
1001: LinkReset \newline 
1010: Reserved \newline 
1011: Retrain \newline 
1100: Disable \newline
\\ \hline
pl\_lnk\_cfg[2:0] & N/A &
Width of the Port: This bit field indicates the width of the port as determined by the Link
initialization: \newline
000 – x1 \newline
001 – x2 \newline
010 – x4 \newline
011 – x8 \newline
100 – x12 \newline 
101 – x16 \newline
110 – x32 \newline

\\ \hline
\end{tabular}
\end{table}


\begin{table}[H]

    \centering
  \begin{tabular}{ |P{30mm}|P{18mm}|m{63mm}|  }
  \hline
pl\_rxframe\_errmask & High & Rx Framing Error Reporting Mask:
When asserted, receiver framing error logging/escalation should be masked off in the
Link Layer. logPHY asserts this based on link state and data path alignment to make
sure false errors are not logged by the Link Layer.
\\ \hline
pl\_speedmode[2:0] & N/A & Current Link Speed as negotiated by the logPHY
(3’b000=Gen1,3’b001=Gen2,3’b010=Gen3,\newline 3’b011=Gen4, 3’b100=Gen5, rest=Rsvd)
Link Layer should only consider this to be relevant when pl\_state\_sts=RETRAIN or
ACTIVE. \\ \hline
pl\_setlabs & High & logPHY’s pulsed indication to set Link Auto Bandwidth Change status in Link Status register \\ \hline
pl\_protocol[2:0] & N/A & logPHY indication to upper layers about which protocol was detected during training. It
has the following encodings: \newline
000b – PCIe \\ \hline
pl\_protocol\_vld & High & 
Indication that pl\_protocol has valid information. This is a level signal, asserted when
the logPHY has discovered the appropriate protocol, but can de-assert again after
subsequent transitions to RESET state depending on the link state machine transitions. \\ \hline
lp\_force\_detect & High & This is a level signal. It forces logPHY to shut down the receiver, drive and keep the
physical LTSSM in Detect. \\ \hline
pl\_phyinrecenter & High & Physical Layer to Link Layer indication that the Physical Layer is in Recovery (Retrain)
state. Please note that pl\_state\_sts indicates the status of the transmitter (training
sequence sent by logPHY for example) whereas this signal is asserted when the receiver
detects recovery entry (training sequences received by the logPHY for example) \\ \hline
\end{tabular}
\end{table}

\begin{table}[H]
    
    \caption{Error signals}
    \centering
  \label{tab:l2}
  \begin{tabular}{ |m{26mm}|m{10mm}|m{60mm}|  }
\hline
\rowcolor{Gray}
\multicolumn{1}{|c|}{\textbf{Name} } 
& \multicolumn{1}{|c|}{\textbf{Active Level}} 
& \multicolumn{1}{|c|}{\textbf{Description}}\\
\hline
pl\_error & High &  
Indicates that the Physical layer detected an encoding or framing related error. This
signal shall be asserted by the Logical PHY when non training related errors are
detected.
Please note that non-training errors are logical PHY specific.
The logging of errors must be determined by the upper level protocols. \\ \hline
pl\_trainerror & High &
Indicates that Physical layer training. Please note that training errors are logical PHY
specific.
The logging of errors must be determined by the upper level protocols. Logical PHY
may use this signal to indicate other uncorrectable errors as well (such as internal parity
errors) and transition to LinkError state as a result.
\\ \hline
pl\_cerror & High  & Indicates that Physical Layer received an error which was corrected by Physical Layer.
The logging of errors must be determined by upper layer protocols. \\ \hline 
lp\_linkerror & High &  
Link Layer to Physical Layer indication that an uncorrectable error has occurred and
Physical Layer must move to LinkError State when it samples this signal. \\ \hline 

\end{tabular}
\end{table}

\begin{table}[H]
    \caption{Clock Gating Interface from logPHY}
    \label{tab:l3}

    \centering
  \begin{tabular}{ |m{26mm}|m{10mm}|m{60mm}|  }
\hline
\rowcolor{Gray}
\multicolumn{1}{|c|}{\textbf{Name} } 
& \multicolumn{1}{|c|}{\textbf{Active Level}} 
& \multicolumn{1}{|c|}{\textbf{Description}}\\
\hline
pl\_exit\_cg\_req & High & When asserted, requests upper layers to exit clock gated state as soon as possible. \\ \hline
lp\_exit\_cg\_ack & High & When asserted, indicates that upper layers are not in clock gated state and are ready to
receive packets from the Physical Layer. \\ \hline
\end{tabular}
\end{table}

\begin{table}[H]
    \caption{Configuration Interface}
    \centering
  \begin{tabular}{ |m{26mm}|m{10mm}|m{60mm}|  }
\hline
\rowcolor{Gray}
\multicolumn{1}{|c|}{\textbf{Name} } 
& \multicolumn{1}{|c|}{\textbf{Active Level}} 
& \multicolumn{1}{|c|}{\textbf{Description}}\\
\hline
pl\_cfg[NC-1:0] & N/A & This is the configuration interface from Logical PHY to the Link Layer. \\ \hline
pl\_cfg\_vld & High & When asserted, indicates that pl\_cfg has valid information that should be
consumed by the Link Layer.
receive packets from the Physical Layer. \\ \hline
lp\_cfg[NC-1:0] & N/A & This is the configuration interface from Link Layer to Logical PHY.\\ \hline
lp\_cfg\_vld & High &When asserted, indicates that lp\_cfg has valid information that should be
consumed by the Logical PHY. \\ \hline
\end{tabular}
\end{table}

\begin{table}[H]
    \caption{Signals for PCIe}
    \centering
  \begin{tabular}{ |m{26mm}|m{10mm}|m{60mm}|  }
\hline
\rowcolor{Gray}
\multicolumn{1}{|c|}{\textbf{Name} } 
& \multicolumn{1}{|c|}{\textbf{Active Level}} 
& \multicolumn{1}{|c|}{\textbf{Description}}\\
\hline
pl\_nbstallreq& High &Physical Layer request to Link Layer to align packets at LPIF width boundary \\ \hline 
lp\_nbstallack& High & Link Layer acknowledge to pl\_nbstallreq.\\ \hline 

pl\_block\_dl\_init &High & Indication from the logPHY to the LL to block initialization of DLLPs.\\ \hline 
lp\_dl\_active &High & Indication from the LL that Data Link Control and Management State Machine is in
DL\_Active state (as defined in the PCIe spec)\\ \hline 
lp\_good\_dllp &High & ndication from Link Layer that a Data Link Layer Packet (DLLP) was received without errors. Used by upstream ports to block DLLP transmission until a good DLLP is received\\ \hline 
pl\_in\_rxl0s & N/A& Indication from logPHY that Receiver is in L0s\\ \hline 
pl\_byte\_err[(n-1):0] &N/A & Corresponding byte of data has an error. In Gen1/2 framing, it denotes k-char error 
detected by logPHY. LL uses this to log an error. It may assert in higher data rates, but it
is not logged by the Link Layer for Gen3 and above speeds.\\ \hline 
&N/A & Corresponding byte of data has an error. In Gen1/2 framing, it denotes k-char error detected by logPHY. LL uses this to log an error. It may assert in higher data rates, but it
is not logged by the Link Layer for Gen3 and above speeds.
\\ \hline 
pl\_kchar[(n-1):0] &N/A & k-char indication from logPHY. When asserted, the corresponding data byte must be
interpreted at the LL as a k-char.\\ \hline 

\end{tabular}
\end{table}

\begin{table}[H]

    \centering
  \begin{tabular}{ |m{26mm}|P{18mm}|m{60mm}|  }
  \hline

pl\_dlpstart[w-1] &N/A & logPHY indicates the start of a Data Link Layer packet for Gen3 and above speeds. Each
bit corresponds to a specific data byte depending on the configuration of the port.\\ \hline 
pl\_dlpend[w-1] &N/A & logPHY indicating the end of a Data Link Layer packet for Gen3 and above speeds. Each
bit corresponds to a specific data byte depending on the configuration of the port.\\ \hline
pl\_tlpstart[w-1] &N/A &logPHY indicating the start of a Transaction Layer packet for Gen3 and above speeds
(STP). \\ \hline
pl\_tlpend[w-1] &N/A & logPHY indicating the end of a Transaction Layer packet for Gen3 and above speeds
(END).\\ \hline
pl\_tlpedb[w-1] &N/A & logPHY indicating EDB received for Gen3 and above speeds. \\ \hline
pl\_rx\_flush &N/A & Request from logPHY to Link Layer to flush its receiver pipeline. This typically occurs for
framing errors in Gen3\\ \hline

\end{tabular}
\end{table}
\section{Internal Signals}

\begin{table}[H]
    \caption{LTSSM(out/in) and Packet Generator Inteface}
    \centering
  \begin{tabular}{ |P{26mm}|m{10mm}|m{60mm}|  }
\hline
\rowcolor{Gray}
\multicolumn{1}{|c|}{\textbf{Name} } 
& \multicolumn{1}{|c|}{\textbf{In/Out}} 
& \multicolumn{1}{|c|}{\textbf{Description}}\\
\hline
ordered\_set\_type\newline[3:0] & out & 

0000 : SOS   \newline 
0001 : EIOS  \newline
0010 : EIEOS  \newline 
0011 : TS1 \newline   
0100 : TS2 \newline 
0101 : FTS \newline
0110 : SDS \newline
0111 : COM \newline
 \\ \hline
Gen\_type & out & Gen $1 \longleftrightarrow 2$:0 
\newline  Gen $3 \longleftrightarrow 5$:1  \\ \hline
link\_number & out & Link Number
\newline -For PAD assign it to 0.
\newline -For default link number assign it to 1.
\\ \hline
N\_FTS [7:0] & out & Number of FTS Ordered Sets required by receiver to achieve L0 when exiting L0s.\\ \hline
lane\_number\newline[3:0] & out & Lane Number 
\newline -Lane number values from 0 to 7. 
\newline -For PAD assign it to 8. 
\newline -From 9 to 15 is Reserved.\\ \hline
rate\_identifier\newline[7:0] & out & Data Rate Identifier: 
\newline -Bit 1: 2.5 GT/s supported (must be set to 1b).
\newline -Bit 2: 5.0 GT/s supported (must be set if bit 3 is set).
\newline -Bit 3: 8.0 GT/s supported.
\newline -Bit 6: Autonomous Change/Selectable De-emphasis.
\newline -Bit 7: Speed change. This can only be set to one in the Recovery.RcvrLock LTSSM state.
\newline -Bits (0,5,4) are Reserved\\ \hline

\end{tabular}
\end{table}


% \begin{table}[H]

%     \centering
%   \begin{tabular}{ |m{26mm}|P{18mm}|m{60mm}|  }
%   \hline
% pl\_rxframe\_errmask & High & Rx Framing Error Reporting Mask:
% When asserted, receiver framing error logging/escalation should be masked off in the
% Link Layer. logPHY asserts this based on link state and data path alignment to make
% sure false errors are not logged by the Link Layer.
% \\ \hline
% pl\_speedmode[2:0] & N/A & Current Link Speed as negotiated by the logPHY
% (3’b000=Gen1,3’b001=Gen2,3’b010=Gen3, 3’b011=Gen4, 3’b100=Gen5, rest=Rsvd)
% Link Layer should only consider this to be relevant when pl_state_sts=RETRAIN or
% ACTIVE. \\ \hline
% pl\_setlabs & High & logPHY’s pulsed indication to set Link Auto Bandwidth Change status in Link Status register \\ \hline
% pl\_protocol[2:0] & N/A & logPHY indication to upper layers about which protocol was detected during training. It
% has the following encodings: \newline
% 000b – PCIe
% \end{tabular}
% \end{table}


% \end{tabular}
% \end{table}

% \begin{table}[H]
%     \caption{TX and Related signals}
%     \centering
%   \begin{tabular}{ |m{26mm}|m{10mm}|m{60mm}|  }
% \hline
% \rowcolor{Gray}
% \multicolumn{1}{|c|}{\textbf{Name} } 
% & \multicolumn{1}{|c|}{\textbf{Active Level}} 
% & \multicolumn{1}{|c|}{\textbf{Description}}\\
% \hline



% \end{tabular}
% \end{table}
