
\chapter{Signal Interface}
\clearpage
\section{PIPE Interface}

\label{sec:1}
\begin{table}[H]
    \caption{TX Related signals}
    \label{tab:p1}
    \centering
  \begin{tabular}{ |m{50mm}|P{5mm}|P{5mm}|m{60mm}|  }
\hline
\rowcolor{Gray}
\multicolumn{1}{|c|}{\textbf{Name} }
&\multicolumn{1}{|c|}{\textbf{I/O} }  
& \multicolumn{1}{|c|}{\textbf{Active Level}} 
& \multicolumn{1}{|c|}{\textbf{Description}}\\
\hline
TxData [MAXPIPEWIDTH* \newline LANESNUMBER-1:0]
& 
OUT
&
N/A
&
Parallel data input bus. \\
\hline

TxDataValid [LANESNUMBER-1:0]
&
OUT 
&
N/A
&
This signal allows the MAC to instruct the PHY to ignore the data interface \newline
for one clock cycle. A value of one \newline
indicates the phy will use the data, a \newline
value of zero indicates the phy will not \newline
use the data. \\
\hline

TxElecIdle[LANESNUMBER-1:0]
&
OUT
& 
High
&
Forces Tx output to electrical idle when asserted
except in loopback.\newline \newline
Note: The MAC must always have TxDataValid
asserted when TxElecIdle transitions to either
asserted or deasserted; TxDataValid is a qualifier
for TxElecIdle sampling.
\\
\hline


TxDataK \newline[(MAXPIPEWIDTH/8)* \newline LANESNUMBER-1:0]
&
OUT
& 
N/A
&
A value of zero indicates a
data byte, a value of 1 indicates a
control byte.\\
\hline


TxStartBlock \newline [LANESNUMBER-1:0]
&
OUT
& 
N/A
&
This signals allow the MAC to tell
the PHY the starting byte for a 128b
block when value is 1. The starting byte for a 128b
block must always start with byte 0 of
the data interface. \\
\hline



TxSyncHeader \newline [2*LANESNUMBER -1:0]
&
OUT
& 
N/A
&

Provides the sync header for
the PHY to use in the next 130b block.
\newline \newline
$ 01 \longrightarrow $control byte \newline
$ 10 \longrightarrow $data byte

\\
\hline


TxDetectRx/ \newline Loopback [LANESNUMBER-1:0]
&
OUT
& 
High
&
Used to tell the PHY to begin a receiver detection
operation or to begin loopback \\
\hline



\end{tabular}

\end{table}


% Rx and related signal
\begin{table}[H]
    \caption{RX Related signals}
    \label{tab:p2}
    \centering
    \begin{tabular}{ |m{50mm}|P{5mm}|P{5mm}|m{60mm}|  }
      \hline
\rowcolor{Gray}
\multicolumn{1}{|c|}{\textbf{Name} } 
& \multicolumn{1}{|c|}{\textbf{I/O} } 
& \multicolumn{1}{|c|}{\textbf{Active Level}} 
& \multicolumn{1}{|c|}{\textbf{Description}}\\
\hline
RxData \newline [MAXPIPEWIDTH* \newline LANESNUMBER-1:0]
&
IN 
&
N/A
&
Parallel data output bus. \\
\hline

RxDataValid \newline [LANESNUMBER-1:0]
&
IN
& 
N/A
&
This signal allows the PHY to instruct the MAC to ignore the data interface \newline
for one clock cycle. A value of one \newline
indicates the mac will use the data, a \newline
value of zero indicates the mac will not \newline
use the data. \\
\hline



RxDataK \newline[(MAXPIPEWIDTH/8)* \newline LANESNUMBER-1:0]
&
IN
& 
N/A
&
A value of zero indicates a
data byte, a value of 1 indicates a
control byte.\\
\hline


RxStartBlock \newline [LANESNUMBER-1:0]
&
IN
& 
N/A
&
This signals allow the PHY to tell
the MAC the starting byte for a 128b
block when value is 1. The starting byte for a 128b
block must always start with byte 0 of
the data interface. \\
\hline



RxSyncHeader \newline [2*LANESNUMBER -1:0]
&
IN
& 
N/A
&

Provides the sync header for
the MAC to use in the next 130b block.
\newline \newline
$ 01 \longrightarrow $control byte \newline
$ 10 \longrightarrow $data byte

\\
\hline


RxValid \newline [LANESNUMBER-1:0]
&
IN
& 
High
&
Indicates symbol lock and valid
data on RxData and RxDataK and
further qualifies RxDataValid
when used.\\
\hline


RxStatus \newline [3*LANESNUMBER -1:0]
&
IN
& 
N/A
&
Encodes receiver status and error
codes for the received data
stream when receiving data.\newline
\begin{tabular}{|m{2mm}|m{2mm}|m{2mm}|m{30mm}| }
 \hline
[0]&[1]&[2]& Description \\
     \hline
0 & 0 & 0 & Received data OK \\
     \hline
0 & 0 & 1 & 1 SKP added \\
     \hline
0 & 1 & 0 & 1 SKP removed \\
     \hline
0 & 1 & 1 & Receiver detected\\
     \hline
1 & 0 & 0 & Both $8B/10B$  $(128B/130B 6 )$ decode 
error and (optionally) 
Receive Disparity error \\
     \hline
1 & 0 & 1 & Elastic Buffer overflow \\
     \hline

1 & 1 & 0 & Elastic Buffer
underflow. \\
     \hline
1 & 1 & 1 & Receive disparity error
(Reserved if Receive
Disparity error is
reported with code
0b100)\\
     \hline

\end{tabular}

\\
\hline


\end{tabular}


\end{table}

\begin{table}[H]

    \centering
  \begin{tabular}{ |m{50mm}|P{5mm}|P{5mm}|m{60mm}|  }
  \hline
RXElecldle & IN & High & Indicates receiver detection of an electrical
idle. While deasserted with the PHY in P2
(PCI Express mode) or the PHY in P0, P1, P2,
or P3 (USB Mode), indicates detection of
either: \newline
PCI Express Mode: a beacon. \\
\hline
\end{tabular}
\end{table}



\begin{table}[H]
    \caption{Clk and reset}
    \centering
  \begin{tabular}{ |m{26mm}|P{10mm}|P{10mm}|m{60mm}|  }
\hline
\rowcolor{Gray}
\multicolumn{1}{|c|}{\textbf{Name} } 
& \multicolumn{1}{|c|}{\textbf{I/O} } 
& \multicolumn{1}{|c|}{\textbf{Active Level}} 
& \multicolumn{1}{|c|}{\textbf{Description}}\\
\hline
 Reset\# & IN & Low & Resets the transmitter and receiver. This signal
is asynchronous. \\
\hline

 CLK & IN & Edge & This differential Input is used to generate
the bit-rate clock for the PHY transmitter
and receiver.\\
\hline

phy\_reset & OUT & Low & Resets phy.\\
\hline


\end{tabular}
\end{table}


\begin{table}[H]
    \caption{Commands and Status signals}
    \label{tab:command}
    \centering
  \begin{tabular}{ |m{26mm}|P{5mm}|P{10mm}|m{60mm}|  }
\hline
\rowcolor{Gray}
\multicolumn{1}{|c|}{\textbf{Name} }
& \multicolumn{1}{|c|}{\textbf{I/O} }
& \multicolumn{1}{|c|}{\textbf{Active Level}} 
& \multicolumn{1}{|c|}{\textbf{Description}}\\
\hline
 
PowerDown[3:0]
 &  OUT & N/A & Power up or down the transceiver. Power states
PCI Express Mode: \newline
\begin{tabular}{|m{2mm}|m{2mm}|m{2mm}|m{2mm}|m{30mm}|}
 \hline
[0]&[1]&[2]&[3]& Description \\
     \hline
     
0& 0 & 0 & 0 & P0, normal operation \\
     \hline
0& 0 & 0 & 1 & P0s, low recovery time
latency, power saving state \\
     \hline

0& 0 & 1 & 0 & P2, lowest power state\\
     \hline

\end{tabular}

\\
\hline

 
Rate[3:0]  
& OUT & N/A & 
 Control the link signaling rate.
 
 \begin{tabular}{|c|c|}
    \hline
     Value & Description  \\ \hline
     0  & Use 2.5 GT/s signaling rate \\ \hline
     1 & Use 5 GT/s signaling rate \\ \hline
     2 & Use 8 GT/s signaling rate\\ \hline
     3 & Use 16 GT/s signaling rate\\ \hline
     4 & Use 32 GT/s signaling rate\\ \hline
 \end{tabular}
\\
\hline


\end{tabular}
\end{table}

\begin{table}[H]

    \centering
  \begin{tabular}{ |m{26mm}|P{5mm}|P{18mm}|m{60mm}|  }
  
\hline
 PhyStatus & IN & High & Used to communicate completion
of several PHY functions including
stable PCLK and/or Max PCLK
(depending on clocking mode)
after Reset\# deassertion, power
management state transitions,
rate change, and receiver
detection.\\
\hline

Width[1:0] & OUT & N/A &Controls the PIPE data path width. \newline 
\begin{tabular}{|c|c|}
\hline
    Value  & Datapath Width  \\ \hline
    0 &  8 bits \\ \hline
    1 & 16 bits \\ \hline
    2 & 32 bits \\ \hline
\end{tabular}
\\
\hline


PclkChangeAck & OUT & High & 
Only used when PCLK is a PHY input.
Asserted by the MAC when a PCLK rate
change or rate change or, if required,
width change is complete and stable. \newline 

After the MAC asserts PclkChangeAck
the PHY responds by asserting
PhyStatus for one cyle and de-asserts
PclkChangeOk at the same time as
PhyStatus. The controller shall deassert
PclkChangeAck when PclkChangeOk is
sampled low.

\\
\hline
PclkChangeOk & IN & High &
Only used when PCLK is a PHY
input. Asserted by the PHY when
it is ready for the MAC to change
the PCLK rate or Rate.
\\
\hline
\end{tabular}
\end{table}



\begin{table}[H]
    \caption{Message Bus Interface Signals}
    \centering
  \begin{tabular}{ |P{30mm}|P{10mm}|m{60mm}|  }
\hline
\rowcolor{Gray}
\multicolumn{1}{|c|}{\textbf{Name} } 
& \multicolumn{1}{|c|}{\textbf{Direction}} 
& \multicolumn{1}{|c|}{\textbf{Description}}\\
\hline

M2P\_MessageBus\newline[7:0] & Input & The MAC multiplexes command, any
required address, and any required data
for sending read and write requests to
access PHY PIPE registers and for
sending read completion responses and
write ack responses to PHY initiated
requests.  \\ \hline 
P2M\_MessageBus\newline[7:0] & Output & The PHY multiplexes command, any
required address, and any required data
for sending read and write requests to
access MAC PIPE registers and for
sending read completion responses and
write ack responses to MAC initiated
requests.  \\ \hline 

\end{tabular}
\end{table}


\begin{table}[H]
  \caption{MAC Interface(in/out) Equalization signals}
  \centering
\begin{tabular}{ |P{60mm}|P{10mm}|m{60mm}|  }
\hline
\rowcolor{Gray}
\multicolumn{1}{|c|}{\textbf{Name} } 
& \multicolumn{1}{|c|}{\textbf{I/O}} 
& \multicolumn{1}{|c|}{\textbf{Description}}\\
\hline

LocalTxPresetCoefficients \newline [18*LANESNUMBER -1:0] & IN & Reads the current equalization values of the transmitter port
PCI Express Mode:
These are the coefficients for the preset on the
LocalPresetIndex[3:0] after a
GetLocalPresetCoeffcients request:
\newline [5:0] C-1
\newline [11:6] C0
\newline  [17:12] C+1
\newline Valid on assertion of LocalTxCoefficientsValid. 
\newline The MAC will reflect these coefficient values
on the TxDeemph bus when MAC wishes to
apply this preset.
These signals are only used by a PHY that
requires dynamic preset coefficient updates.
\\
\hline

TXDeemph \newline [18*LANESNUMBER -1:0]  & OUT & 
When rate is 2.5 or 5.0 GT/s
\begin{tabular}{|c|c|}
  \hline
      Value  & Description  \\ \hline
      0 & -6dB de-emphasis \\ \hline
      1 & -3.5dB de-emphasis \\ \hline
      2 & No de-emphasis\\ \hline
      3 & Reserved\\ \hline
  \end{tabular} \newline
PIPE implementations that only support 2.5
GT/s do not implement this signal. PIPE PHY
implementations that do not support low swing
are not required to support the no-de-
emphasis mode.
When the rate is 8.0 GT/s
\newline [5:0]
C -1
\newline [11:6]
C 0
\newline [17:12]
C +1 \newline 
Note: The MAC must ensure that only
supported values are used for TxDeemph.
\\
\hline

LocalFS \newline [6*LANESNUMBER -1:0] & IN & 
PCI Express Mode: \newline
Provides the FS value for the PHY. These
signals are only used by a PHY that requires
dynamic preset coefficient updates. \newline
This value shall be sampled by the MAC only
when PhyStatus is pulsed after RESET or on
the first PhyStatus pulse after a rate change to
8 GT/s.
\\ \hline

LocalLF \newline [6*LANESNUMBER -1:0] & IN & 

PCI Express Mode: \newline 
Provides the LF value for the PHY. This
signal is only used by a PHY that requires
dynamic preset coefficient updates. 
This value must only be sampled by the MAC
only when PhyStatus is pulsed after RESET
or on the first PhyStatus pulse after a rate
change to 8 GT/s.
\\ \hline




\end{tabular}
\end{table}


\begin{table}[H]

  \centering
\begin{tabular}{ |P{60mm}|P{10mm}|m{60mm}| }

\hline
GetLocalPresetCoeffcients [LANESNUMBER -1:0] & OUT &
PCI Express Mode: \newline
A MAC holds this signal high for one PCLK
cycle requesting a preset to co-efficient
mapping for the preset on
LocalPresetIndex[3:0] to coefficients on
LocalTxPresetCoefficient[17:0]
Maximum Response time of PHY is 128 nSec.
Note. A MAC can make this request any time
after reset.\newline
Note. After a local preset coefficient request a
MAC could assert GetLocalPresetCoeffcients
again as soon as the next PCLK after
LocalTxCoefficientsValid deasserts.
\newline
This signal is only used with a PHY that
requires dynamic preset coefficient updates

\\ \hline
LocalTxCoefficientsValid [LANESNUMBER -1:0] & IN & 
PCI Express Mode: \newline
A PHY holds this signal high for one PCLK
cycle to indicate that the
LocalTxPresetCoefficients[17:0] bus correctly
represents the coefficients values for the
preset on the LocalPresetIndex bus. \newline
This signal is only used by a PHY that requires
dynamic preset coefficient updates

\\ \hline

LF [LANESNUMBER -1:0] & OUT & 
PCI Express Mode: \newline
Provides the LF value advertised by the link
partner. The MAC shall only change this value
when a new LF value is captured during link
training. A PHY may optionally consider this
value when deciding how long to evaluate TX
equalization settings of the link partner.
These signals are only used at the 8.0 GT/s
signaling rate
\\ \hline
RxEqEval [LANESNUMBER -1:0] & OUT & 
PCI Express Mode: \newline
The PHY starts evaluation of the far end
transmitter TX EQ settings when the MAC asserts
this signal.
This signal is only used at the 8.0 GT/s
signaling rate.
\\ \hline
\end{tabular}
\end{table}

\begin{table}[H]

  \centering
\begin{tabular}{ |P{60mm}|P{10mm}|m{60mm}| }
\hline
LocalPresetIndex \newline [4*LANESNUMBER -1:0] & OUT & 
PCI Express Mode: \newline
Index for local PHY preset coefficients
requested by the MAC \newline
The preset index value is encoded as follows: \newline
0000b – Preset P0.    \newline
0001b – Preset P1.  \newline
0010b – Preset P2.  \newline
0011b – Preset P3.  \newline
0100b – Preset P4.  \newline
0101b – Preset P5.  \newline
0110b – Preset P6.  \newline
0111b – Preset P7.  \newline
1000b – Preset P8.  \newline
1001b – Preset P9.  \newline
1010b – Preset P10. \newline
1011b – Reserved      \newline
1100b – Reserved      \newline
1101b – Reserved    \newline
1110b – Reserved      \newline
1111b – Reserved.     \newline
These signals are only      \newline
These signals are only used with a PHY that
requires dynamic preset coefficient updates.
\\ \hline
\end{tabular}
\end{table}


\begin{table}[H]

    \centering
  \begin{tabular}{ |P{60mm}|P{10mm}|m{60mm}| }
    
  \hline

InvalidRequest [LANESNUMBER -1:0] & OUT & 
PCI Express Mode: \newline
Indicates that the Link Evaluation feedback
requested a link partner TX EQ setting that was out
of range. The MAC asserts this signal when it
detects an out of range error locally based on
calculated link partner transmitter coefficients
based on the last valid link equalization feedback
or it receives a NACK response from the link
partner. The MAC keeps the sigal asserted until
the next time it asserts RxEQEval. When a MAC
asserts this signal it shall subsequently ask the
PHY to perform an RxEQ evaluation using the last
valid setting a second time.
This signal is only used at the 8.0 GT/s
signaling rate.
\\ \hline
  LinkEvaluationFeedbackDirectionChange \newline  [6*LANESNUMBER -1:0]  & IN & 
  PCI Express Mode: \newline
  Provides the link equalization evaluation feedback
  in the direction change format. Feedback is
  provided for each coefficient:\newline
  [1:0] C -1\newline
  [3:2] C 0\newline
  [5:4] C 1\newline
  The feedback value for each coefficient is encoded
  as follows:\newline
  00 - No change\newline
  01 – Increment\newline
  10 – Decrement\newline
  11 - Reserved\newline
  A PHY does not implement these signals if it is
  does not provide link equalization evaluation
  feedback using the Direction Change format.
  Note: In 8.0 GT/s mode the MAC shall ignore the
  C 0 value and use the correct value per the PCI
  Express specification.\newline
  These signals are only used at the 8.0 GT/s
  signaling rate.\newline
  Note that C -1 and C 1 are encoded as the absolute
  value of the actual FIR coefficient and thus
  incrementing or decrementing either value refers to
  the magnitude of the actual FIR coefficient.
  For example – if C -1 is 000001b the FIR coefficient
  is negative one and a request to increment C -1 will
  increase it in the direction of 000002b which
  decreases the FIR coefficient.
  \\ \hline
\end{tabular}
\end{table}


\section{LPIF Inteface}
\label{sec:2}
\begin{table}[H]
\label{tab:l1}
    \caption{LPIF Interface Signals}
    \centering
  \begin{tabular}{ |P{26mm}|P{5mm}|P{10mm} | m{60mm}|  }
\hline
\rowcolor{Gray}
\multicolumn{1}{|c|}{\textbf{Name} } 
& \multicolumn{1}{|c|}{\textbf{I/O} } 

& \multicolumn{1}{|c|}{\textbf{Active Level}}
& \multicolumn{1}{|c|}{\textbf{Description}}
\\
\hline

pl\_trdy & OUT & High & LIndicates Physical Layer is ready to accept data. Data is accepted when pl\_trdy, lp\_valid, and lp\_irdy are asserted together.\\ \hline


lp\_irdy & IN & High & Link Layer to Physical Layer indicating Link Layer is ready to transfer data. lp\_irdy must not be presented by the upper layers when pl\_state\_sts is RESET.\\ \hline


lp\_valid [64 -1:0] & IN & High & Link Layer to Physical Layer indicates data valid on the corresponding lp\_data bytes.
‘LP\_NVLD’ equals the number of valid bits. The bytes of lp\_data associated with a
specific bit of lp\_valid is implementation specific. When lp\_irdy is asserted, at least one
of the bits of lp\_valid must be asserted.\\ \hline
pl\_data [512 -1:0] & OUT & N/A &
Physical Layer to Link Layer Data, where ‘NBYTES’ equals number of bytes determined
by the supported data bus for the LPIF interface. \\ \hline
pl\_valid [64 -1:0] & OUT & High &
Physical Layer to Link Layer indicates data valid on pl\_data. ‘PL\_NVLD’ equals the
number of valid bits. The bytes of pl\_data associated with a specific bit of pl\_valid is
implementation specific.\\ \hline

lp\_data[512 -1:0] & IN & N/A &
Link Layer to Physical Layer Data,
where `NBYTES' equals number of
bytes determined by the data width for
the LPIF instance. \\ \hline

\end{tabular}
\end{table}

\begin{table}[H]

    \centering
  \begin{tabular}{ |P{26mm}|P{5mm}|P{10mm} | m{60mm}|  }
  \hline

\hline
lp\_state\_req[3:0] & IN & N/A & 
Link Layer Request to Logical Physical Layer to request state change.
Encodings as follows: \newline
0000: NOP \newline 
0001: Active \newline 
0010: Active.L0s \newline 
0011: Deepest Allowable PM State [L1 Substates only] \newline 
0100: L1.1 \newline 
0101: L1.2 \newline 
0110: L1.3 \newline 
0111: L1.4 \newline 
1000: L2 \newline 
1001: LinkReset \newline 
1010: Reserved \newline 
1011: Retrain \newline 
1100: Disable \newline 
\\ \hline
pl\_state\_sts[3:0] & OUT & N/A &
Physical Layer to Link Layer Status indication of the Interface.
Encodings as follows:
\newline
0000: NOP \newline 
0001: Active \newline 
0010: Active.L0s \newline 
0011: Deepest Allowable PM State [L1 Substates only] \newline 
0100: L1.1 \newline 
0101: L1.2 \newline 
0110: L1.3 \newline 
0111: L1.4 \newline 
1000: L2 \newline 
1001: LinkReset \newline 
1010: Reserved \newline 
1011: Retrain \newline 
1100: Disable \newline
\\ \hline
pl\_speedmode[2:0] & OUT & N/A & Current Link Speed as negotiated by the logPHY
(3’b000=Gen1,3’b001=Gen2,3’b010=Gen3,\newline 3’b011=Gen4, 3’b100=Gen5, rest=Rsvd)
Link Layer should only consider this to be relevant when pl\_state\_sts=RETRAIN or
ACTIVE. \\ \hline
lp\_force\_detect & IN & High & This is a level signal. It forces logPHY to shut down the receiver, drive and keep the
physical LTSSM in Detect. \\ \hline
\end{tabular}
\end{table}



\begin{table}[H]

    \centering
  \begin{tabular}{ |m{26mm}|P{10mm}|P{18mm}|m{60mm}|  }
  \hline

pl\_dlpstart[64-1] & OUT &N/A & logPHY indicates the start of a Data Link Layer packet for Gen3 and above speeds. Each
bit corresponds to a specific data byte depending on the configuration of the port.\\ \hline 
pl\_dlpend[64-1]& OUT &N/A & logPHY indicating the end of a Data Link Layer packet for Gen3 and above speeds. Each
bit corresponds to a specific data byte depending on the configuration of the port.\\ \hline
pl\_tlpstart[64-1]& OUT &N/A &logPHY indicating the start of a Transaction Layer packet for Gen3 and above speeds
(STP). \\ \hline
pl\_tlpend[64-1]& OUT &N/A & logPHY indicating the end of a Transaction Layer packet for Gen3 and above speeds
(END).\\ \hline
pl\_tlpedb[64-1]& OUT &N/A & logPHY indicating EDB received for Gen3 and above speeds. \\ \hline

lp\_dlpstart[64-1] & IN &N/A & 
Indicates from Link Layer start of dlp\\ \hline 
lp\_dlpend[64-1]& IN &N/A & 
Indicates from Link Layer end of dlp\\ \hline
lp\_tlpstart[64-1]& IN &N/A &
Indicates from Link Layer start of tlp \\ \hline
lp\_tlpend[64-1]& IN &N/A & 
Indicates from Link Layer end of tlp\\ \hline
lp\_tlpedb[64-1]& IN &N/A & 
Indicates from Link Layer end o of bad tlp\\ \hline

\end{tabular}
\end{table}
% \section{Internal Signals}

% \begin{table}[H]
%     \caption{LTSSM(in/out) and ordered set Inteface}
%     \centering
%   \begin{tabular}{ |P{26mm}|m{10mm}|m{60mm}|  }
% \hline
% \rowcolor{Gray}
% \multicolumn{1}{|c|}{\textbf{Name} } 
% & \multicolumn{1}{|c|}{\textbf{In/Out}} 
% & \multicolumn{1}{|c|}{\textbf{Description}}\\
% \hline
% ordered\_set\_type\newline[3:0] & out & 

% 0000 : SOS   \newline 
% 0001 : EIOS  \newline
% 0010 : EIEOS  \newline 
% 0011 : TS1 \newline   
% 0100 : TS2 \newline 
% 0101 : FTS \newline
% 0110 : SDS \newline
% 0111 : COM \newline
%  \\ \hline
% Gen\_type & out & Gen $1 \longleftrightarrow 2$:0 
% \newline  Gen $3 \longleftrightarrow 5$:1  \\ \hline
% link\_number & out & Link Number
% \newline -For PAD assign it to 0.
% \newline -For default link number assign it to 1.
% \\ \hline
% Eq\_info[7:0] & out & equalization info is needded by 
% TS1 and TS2  \\ \hline
% lane\_number\newline[3:0] & out & Lane Number 
% \newline -Lane number values from 0 to 7. 
% \newline -For PAD assign it to 8. 
% \newline -From 9 to 15 is Reserved.\\ \hline
% rate\_identifier\newline[7:0] & out & Data Rate Identifier: 
% \newline -Bit 1: 2.5 GT/s supported (must be set to 1b).
% \newline -Bit 2: 5.0 GT/s supported (must be set if bit 3 is set).
% \newline -Bit 3: 8.0 GT/s supported.
% \newline -Bit 6: Autonomous Change/Selectable De-emphasis.
% \newline -Bit 7: Speed change. This can only be set to one in the Recovery.RcvrLock LTSSM state.
% \newline -Bits (0,5,4) are Reserved\\ \hline

% \end{tabular}
% \label{or}
% \end{table}
% \begin{table}[H]
%   \caption{LTSSM(in/out) and ordered set decoder}
%   \centering
% \begin{tabular}{ |P{50mm}|P{10mm}|m{40mm}|  }
% \hline
% \rowcolor{Gray}
% \multicolumn{1}{|c|}{\textbf{Name} } 
% & \multicolumn{1}{|c|}{\textbf{In/Out}} 
% & \multicolumn{1}{|c|}{\textbf{Description}}\\
% \hline

% ordered\_set\_type\_Detector[3:0]  \newline \newline
%  Gen\_type\_Detector \newline \newline
% link\_number\_Detector \newline  \newline
% Eq\_info\_Detector[7:0] \newline \newline
% lane\_number\_Detector[3:0] \newline \newline
% rete\_identifier\_Detector[7:0] 
% &in
% & Decode the information get from ordered set in table \ref{or}
% \\ \hline
% \end{tabular}
% \end{table}

% % \begin{table}[H]
% %   \caption{LTSSM(in/out) and LPIF Inteface}
% %   \centering
% % \begin{tabular}{ |P{26mm}|P{10mm}|m{60mm}|  }
% % \hline
% % \rowcolor{Gray}
% % \multicolumn{1}{|c|}{\textbf{Name} } 
% % & \multicolumn{1}{|c|}{\textbf{In/Out}} 
% % & \multicolumn{1}{|c|}{\textbf{Description}}\\
% % \hline

% % a & b & c \newline d \\ \hline

% % \end{tabular}
% % \end{table}

% \begin{table}[H]
%   \caption{LTSSM(in/out) and clock management}
%   \centering
% \begin{tabular}{ |P{30mm}|P{10mm}|m{60mm}|  }
% \hline
% \rowcolor{Gray}
% \multicolumn{1}{|c|}{\textbf{Name} } 
% & \multicolumn{1}{|c|}{\textbf{In/Out}} 
% & \multicolumn{1}{|c|}{\textbf{Description}}\\
% \hline

% PclkChangeAckM2P & in & acknowlede LTSSM that finish changing rate. \newline
% for more information see table \ref{tab:command} \\ \hline
% PCLK Rate[4:0]  & out & change pclk rate. \newline
% for more information see table \ref{tab:command}  \\ \hline
% \end{tabular}
% \end{table}


% \begin{table}[H]
%   \caption{LTSSM(in/out) and Timer}
%   \centering
% \begin{tabular}{ |P{26mm}|P{10mm}|m{60mm}|  }
% \hline
% \rowcolor{Gray}
% \multicolumn{1}{|c|}{\textbf{Name} } 
% & \multicolumn{1}{|c|}{\textbf{In/Out}} 
% & \multicolumn{1}{|c|}{\textbf{Description}}
% \\
% \hline

% set\_timer[2:0] & out & Set the timer interval to value predefined 
% \newline       001 mean 12 ms
% \newline       010 mean 24 ms 
% \\ \hline

% start\_timer & out & Signal to turn the timer ON
% \\ \hline

% time\_out\_flag & in & Signal to indicate the time interval specified earlier has finished
% \\ \hline



% \end{tabular}
% \end{table}

% \begin{table}[H]
%   \caption{PIPE interface(in/out) and PIPE register}
%   \centering
% \begin{tabular}{ |P{26mm}|P{10mm}|m{60mm}|  }
% \hline
% \rowcolor{Gray}
% \multicolumn{1}{|c|}{\textbf{Name} } 
% & \multicolumn{1}{|c|}{\textbf{In/Out}} 
% & \multicolumn{1}{|c|}{\textbf{Description}}
% \\
% \hline

% write\_reg[7:0]] & out & write registers in PIPE Register
% \\ \hline

% read\_reg[7:0] & out & read registers in PIPE Register
% \\ \hline

% \end{tabular}
% \end{table}
% \begin{table}[H]

%     \centering
%   \begin{tabular}{ |m{26mm}|P{18mm}|m{60mm}|  }
%   \hline

% \end{tabular}
% \end{table}


% \end{tabular}
% \end{table}

% \begin{table}[H]
%     \caption{TX and Related signals}
%     \centering
%   \begin{tabular}{ |m{26mm}|m{10mm}|m{60mm}|  }
% \hline
% \rowcolor{Gray}
% \multicolumn{1}{|c|}{\textbf{Name} } 
% & \multicolumn{1}{|c|}{\textbf{Active Level}} 
% & \multicolumn{1}{|c|}{\textbf{Description}}\\
% \hline



% \end{tabular}
% \end{table}
