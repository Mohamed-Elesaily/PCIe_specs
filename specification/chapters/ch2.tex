\chapter{Features}
In this chapter, we will discuss  the states that we are going to support, design and implement.
% \section{Limitations}
% \begin{table}[H]
%     \caption{unsupported Blocks}
%     \centering
%   \begin{tabular}{ |m{26mm}|}
% \hline
% \rowcolor{Gray}
%  \multicolumn{1}{|c|}{\textbf{Unsupported Blocks}} 
% \\
% \hline


% 8/10b encoder \\ \hline 

% 128/130b encoder \\ \hline
% Serializer/ De-serializer \\ \hline
% Differential driver and receiver \\ \hline
% Elastic buffer logic \\ \hline 
% PLL (phase locked loop)
%  \\ \hline

% Lane de-skew delay circuit \\ \hline
% Block alignment and detect logic
%  \\ \hline
% Clock compensation \\ \hline
% \end{tabular}
% \end{table}
% \begin{table}[H]
%     \caption{unsupported states}
%     \centering
%   \begin{tabular}{ |m{26mm}|}
% \hline
% \rowcolor{Gray}
%  \multicolumn{1}{|c|}{\textbf{Unsupported States}} 
% \\
% \hline

% L1 \\ \hline 
%  L2 \\ \hline
% Hot Reset \\ \hline
% compliance  \\ \hline
% Loopback \\ \hline 
% Disable \\ \hline
% \end{tabular}

% \end{table}
\section{Supported Features}
\begin{table}[H]
    \caption{supported states}
    \centering
  \begin{tabular}{ |m{26mm}|}
\hline
\rowcolor{Gray}
\multicolumn{1}{|c|}{\textbf{
Supported States
} } \\
\hline

Detect \\ \hline 
Polling \\ \hline
Configuration\\ \hline
Recovery\\ \hline
L0\\ \hline 

\end{tabular}

\end{table}





The LTSSM is composed of the following 11 states: Detect, Polling, Configuration, Recovery, L0, L0s, L1, L2, Hot Reset, Loopback, Disable. So here is a quick overview about the supported states:
\begin{enumerate}
    \item Detect \newline 
    It is the initial state of the physical layer, only used at Gen1 2.5 GT/s rate, or converted from the data link layer, or after reset, or from other states (Disable, Polling, Configuration, Recovery, etc.) Conversion. In short, the Detect state is the beginning of PCIe link training. In addition, Detect, as the name suggests, needs to implement detection work. Because in this state, the transmitting end TX needs to detect whether the receiving end RX exists and can work normally, if the detection is normal, it can enter other states. The Detect state mainly includes two sub-states: Detect.Quiet and Detect.Active.
    \item  Polling \newline 
    The purpose of this state is to "pair the cipher" and achieve barrier-free communication. After entering this state, the TS1 and TS2 OS sequences are sent between TX and RX to determine Bit Lock, Symbol Lock and solve the problem of Lane polarity reversal. The Polling state mainly includes three sub-states: Polling.Active, Polling.Configuration, Polling.Compliance.
    \item Configuration \newline 
    The content of this state is very simple. It is to determine the Link/Lane number by sending TS1 and TS2. Configuration contains 6 sub-states: Configuration.LinkWidth.Start, Configuration.LinkWidth.Accpet, Configuration.Lanenum.Wait, Configuration. Lanenum.Accpet, Configuration.Complete, Configuration.Idle

    \item Recovery \newline 
    When the PCIe link needs to be retrained, it enters the Recovery state. There are mainly the following situations:
\begin{itemize}
    \item When the PCIe link signal finds an error, the Bit Lock and Symbol Lock need to be adjusted

    \item  Exit from L0s state
    \item Speed Change. Because the first time you enter the L0 state, the rate is 2.5GT/s. When you need to adjust the rate to 5.0GT/s or 8.0GT/s, you need to enter the Recovery state for Speed Change. At this stage, Bit Lock, Symbol Lock, etc. Need to reacquire
    \item Need to readjust the Width of PCIe link
    \item Need to readjust the Width of PCIe link
    \item Only in Gen3 and Gen4, Equalization needs to be performed again.
\end{itemize}
    \item L0 \newline 
    This is the normal state (Normal and Full-Active State) of the link, and all TLP, DLLP and Ordered Sets can be sent and received normally. In this state, the rate can be 2.5GT/s or 5GT/s or higher (if the devices at both ends of the link support it and have undergone Re-Training).
    

\end{enumerate}
% \begin{table}[H]
%     \caption{TX and Related signals}
%     \centering
%   \begin{tabular}{ |m{26mm}|m{10mm}|m{60mm}|  }
% \hline
% \rowcolor{Gray}
% \multicolumn{1}{|c|}{\textbf{Name} } 
% & \multicolumn{1}{|c|}{\textbf{Active Level}} 
% & \multicolumn{1}{|c|}{\textbf{Description}}\\
% \hline



% \end{tabular}
% \end{table}
\section{Unsupported States}
\begin{table}[H]
    \caption{unsupported states}
    \centering
  \begin{tabular}{ |m{50mm}|}
\hline
\rowcolor{Gray}
\multicolumn{1}{|c|}{\textbf{
Unsupported States
} } \\
\hline

Power managemen states (L1,L2,L0s) \\ \hline 
HotReset \\ \hline
External loopback\\ \hline
Disabled\\ \hline

\end{tabular}

\end{table}


