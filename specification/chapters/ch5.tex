\chapter{Scenarios}

\section{TX Scenarios}
\subsection{Sending TLP or DLLP}
\begin{enumerate}
\item Data (TLP or DLLP) comes from Data Link Layer and is stored in the FIFO.
\item In GEN 1\&2 FIFO Appends STP (in case of TLP), SDP (in case of DLLP) and END Symbols to TLP or DLLP and sends to Mux data, data valid and d/k signal .
\item In Gen 3\&4\&5 FIFO Appends STP token (in case of TLP), SDP token (in case of DLLP) and EDS token. 
\item In GEN 3\&4\&5, if there is an order set (except skip order set) after data stream, FIFO should append EDS token. 
\item Tx LTSSM states that we are in L0 state.
\item Tx LTSSM informs the MUX to take the output coming from the FIFO.
\item Tx LTSSSM informs the Lane Management Control that data coming from the output of MUX is TLP or DLLP.
\item Data is scrambled using scrambler.
\item In GEN 1\&2 Pipe Tx Data outputs the data and assert data valid \& datak. 
\item In GEN 3\&4\&5, Pipe Tx Data outputs the data and assert data valid and adds sync header and start of block before sending data to the PIPE interface.
\end{enumerate}
\subsection{Sending Ordered Sets}
\begin{enumerate}
  \item Tx LTSSM based on current state will send to os generator the Type of order set it needs to send, OS fields, and count of how many times it will send it.
  \item Tx LTSSM sends to FIFO to hold the data inside it.
  \item Tx LTSSM informs the MUX to take the OS generator output.
  \item Tx LTSSM informs Lane Management Control that data coming from the output of MUX is order set.
  \item Link management control should distribute the OS on all lanes as per spec.
  \item All order sets bypass scrambling except for symbols from 1 to 15 for TS order sets in GEN 3\&4\&5.
  \item In GEN 1\&2, Pipe Tx Data outputs the OS and assert data valid \& datak. 
  \item In GEN 3\&4\&5, Pipe Tx Data outputs the OS and assert data valid and adds sync header and start of block before sending OS to the PIPE interface. 
\end{enumerate}

\section{RX Scenario}
\subsection{Receiving TLP or DLLP}
\begin{enumerate}
  \item PIPE Rx data checks data valid is asserted or not.
  \item For GEN 3\&4\&5, PIPE Rx data checks for the sync header to know whether packet received is data (TLP or DLLP) or order set.
  \item Descrambles the data using descrambler.
  \item Link control management un-stripes data bytes.
  \item In GEN 1\&2 Packet identifier will take the output of the link management Control and will remove STP (in case of TLP), SDP (in case of DLLP) and END and set sop and eop.
  \item In GEN 3\&4\&5 Packet identifier will take the output of the link management Control and will remove STP token (in case of TLP), SDP token (in case of DLLP) 
  \item In GEN 3\&4\&5, in case of DLLP after removing SDP token we will forward the next 6 symbols to the FIFO.
  \item In GEN 3\&4\&5, in case of TLP after removing STP token we will forward the next (length – 1) x 4 symbols to the FIFO.
  \item FIFO forwards data to datalink layer.
\end{enumerate}
\subsection{Receiving Ordered Sets}
\begin{enumerate}
  \item Order set come on all lanes simultaneously.
  \item PIPE Rx data checks data valid is asserted or not.
  \item For GEN 3\&4\&5, PIPE Rx data checks for the sync header to know whether packet received is data (TLP or DLLP) or order set.
  \item All order sets bypass descrambling except for symbols from 1 to 15 for TS order sets in GEN 3\&4\&5.
  \item Link control management un-stripes data bytes.
  \item OS Decoder will decode the order set and give its information to Rx LTSSM
\end{enumerate}
\subsection{Linkup scenario}

\subsection{Detect state}
\begin{enumerate}
  \item Detect.quiet 
 \newline \textbf{Transmitter side}
  \begin{itemize}
    \item Tx LTSSM inform PIPE TX Ctrl to set TxElecIdle signal 
    \item Tx LTSSM inform PIPE TX Ctrl to output Rate signal to GEN1 (2.5 G) 
    \item After 12 ms timeout EXIT TO Detect.active.
  \end{itemize}
    \textbf{Receiver side}
\begin{itemize}
  
  \item If any lane exit Electrical Idle
   (RxElecIdle is low) EXIT TO Detect.active.
\end{itemize}
  \item Detect.active
 \textbf{Transmitter side}  
 \begin{itemize}
   \item  TX LTSSM inform PIPE TX Ctrl to start detect process
  \item   If  PIPE TX Ctrl inform Receiver detection 
\begin{itemize}
  \item TX LTSSM inform PIPE TX Ctrl to unset TxElecIdle signal 
  \item TX LTSSM EXIT TO Polling state
\end{itemize}
\item Else After 12 ms timeout EXIT TO Detect.active.

 \end{itemize}
        
\textbf{Receiver side}  
\begin{itemize}
  \item N/A
\end{itemize}    

\end{enumerate}
\subsection{Polling state}

    Polling.Active
  \newline
  \textbf{Transmitter Side}
  \begin{enumerate}
    \item Tx LTSSM will inform the OS 
    generator to generate 1024 TS1s
     on all detected lanes$ \rightarrow$
      filling the Lane and Link Number
       fields with PAD Symbol, advertising
        all the data rates that the device
         supports and setting the Loopback bit of Symbol 5 to 1b.
    \item Tx LTSSM will inform FIFO to hold the data inside it.
    \item Tx LTSSM will inform the MUX to pass TS1s Ordered Set.
    \item Tx LTSSM informs Lane Management Control that the output of the MUX is ordered set, so LMC will strip the TS1s on all the lanes simultaneously.
    \item  These TS1s will bypass the Scrambler (except for Symbols 1 to 15 in GEN 3\&4\&5) and then out to the PHY through PIPE Tx data.

  \end{enumerate}
  

  \textbf{Receiver Side}
  \begin{enumerate}
    
   \item The PIPE Rx Data checks whether data valid is asserted or not.
   \item For GEN 3\&4\&5, PIPE Rx Data checks for the Sync Header to know whether it is Ordered Set or data packet.
   \item TS1s bypass De-scrambler except for Symbols 1 to 15 in GEN 3\&4\&5. 
   \item LMC un-stripes the TS1s. 
   \item OS Decoder will decode these TS1s and give their information to Rx LTSSM $ \rightarrow$ all detected lanes should receive 8 consecutive TS1s with the Lane and Link Number fields filled with PAD Symbol and setting the Loopback bit of Symbol 5 to 1b.

  \end{enumerate}
 Polling.Configuration
  \newline \textbf{Transmitter Side}
  \begin{enumerate}
    
    \item After receiving one TS2 on the receiver side, Tx LTSSM will inform the OS generator to generate at least 16 TS2s $ \rightarrow$ filling the Lane and Link Number fields with PAD Symbol and advertising all the data rates that the device supports.
    \item Tx LTSSM will inform FIFO to hold the data inside it.
    \item Tx LTSSM will inform the MUX to pass TS2s Ordered Set.
    \item Tx LTSSM informs Lane Management Control that the output of the mux is ordered set, so LMC will strip the TS2s on all the lanes simultaneously.
    \item  These TS2s will bypass the Scrambler (except for Symbols 1 to 15 in GEN 3\&4\&5) and then out to the PHY through PIPE Tx data.

  \end{enumerate} 
  \textbf{Receiver Side}
  \begin{enumerate}
    
    \item The PIPE Rx Data checks whether data valid is asserted or not.
    \item For GEN 3\&4\&5, PIPE Rx Data checks for the Sync Header to know whether it is Ordered Set or data packet.
    \item TS2s bypass De-scrambler except for Symbols 1 to 15 in GEN 3\&4\&5. 
    \item LMC un-stripes the TS2s. 
    \item OS Decoder will decode these TS2s and give their information to Rx LTSSM $ \rightarrow$ on any Lane, 8 consecutive TS2s should be received with the Lane and Link Number fields filled with PAD Symbol.
  \end{enumerate}

Configuration state:\newline
\begin{enumerate}
\item LinkWidth.Start
\newline\textbf{Transmitter Side}
 \begin{enumerate}
  \item (Downstream) Tx LTSSM will specify link numbers and PAD lane numbers then pass it to OS generator to generate TS1 ordered set.
  \item (Upstream) Tx LTSSM will send same link numbers received in the receiver side and PAD lane numbers then pass it to OS generator to generate TS1 ordered set.
  \item Tx LTSSM will inform the MUX to pass the TS1OS.
  \item Tx LTSSM will inform LMC to distribute the TS1OS on all lanes simultaneously.
  \item The TS1OS will pass through Scrambler then out to the PHY through the PIPE Tx data.
\end{enumerate}
\textbf{Receiver Side}
\begin{enumerate}
  \item The PIPE Rx Data checks data valid is asserted or not.
  \item For GEN 3\&4\&5, PIPE Rx Data checks for the sync header to know whether packet received is data (TLP or DLLP) or order set.
  \item All order sets bypass Descrambler except for symbols from 1 to 15 for TS ordered sets in GEN 3\&4\&5.
  \item The LMC un-stripes the data.
  \item The OS Decoder will decode the TS1OS and give its information to Rx LTSSM.
  \item (Upstream) Rx LTSSM takes the link and lanes numbers then it forwards it to Tx LTSSM through Main LTSSM to generate TS1OS with same values to send it to the Downstream port and after receiving 2 TS1OS with same link and lane numbers the LTSSM goes to Configuration.LinkWidth.Accept substate.
  \item (Downstream) Upon Rx LTSSM receives 2 TS1OS with same link and lane numbers the LTSSM goes to Configuration.LinkWidth.Accept substate.
\end{enumerate}
\item LinkWidth.Accept
\newline\textbf{Transmitter Side}
\begin{enumerate}
  \item (Downstream) Tx LTSSM initiates lane numbers with same link number then pass it to OS generator to generate TS1 ordered set.
  \item (Upstream) Tx LTSSM will send its lane numbers and same link number then pass it to OS generator to generate TS1 ordered set.
  \item Tx LTSSM will inform the MUX to pass the TS1OS.
  \item Tx LTSSM will inform LMC to distribute the TS1OS on all lanes simultaneously.
  \item The TS1OS will pass through Scrambler then out to the PHY through the PIPE Tx data.
\end{enumerate}
\textbf{Receiver Side}
\begin{enumerate}
  \item The PIPE Rx Data checks data valid is asserted or not.
  \item For GEN 3\&4\&5, PIPE Rx Data checks for the sync header to know whether packet received is data (TLP or DLLP) or order set.
  \item All order sets bypass Descrambler except for symbols from 1 to 15 for TS ordered sets in GEN 3\&4\&5.
  \item The LMC un-stripes the data.
  \item The OS Decoder will decode the TS1OS and give its information to Rx LTSSM.
  \item (Upstream) Rx LTSSM takes the link number and forward it to Tx LTSSM through Main LTSSM to generate TS1OS with same link numbers and with its lane numbers to send it to the Downstream port the LTSSM goes to Configuration.Lanenum.Wait substate.
  \item (Downstream) Doesn’t wait to receive TS1OS as after sending TS1OS with non-PAD link and lane numbers the LTSSM goes to Configuration.Lanenum.Wait substate.
\end{enumerate}
\item Configuration.Lanenum.Wait
\newline\textbf{Transmitter Side}
\begin{enumerate}
  \item (Downstream) Tx LTSSM keeps the same lane numbers and link number as in previous states and pass it to OS generator to generate TS1 ordered set.
  \item (Upstream) Tx LTSSM will keep sending its lane numbers and same link number then pass it to OS generator to generate TS1 ordered set.
  \item Tx LTSSM will inform the MUX to pass the TS1OS.
  \item Tx LTSSM will inform LMC to distribute the TS1OS on all lanes simultaneously.
  \item The TS1OS will pass through Scrambler then out to the PHY through the PIPE Tx data.
\end{enumerate}
\textbf{Receiver Side}
\begin{enumerate}
  \item The PIPE Rx Data checks data valid is asserted or not.
  \item For GEN 3\&4\&5, PIPE Rx Data checks for the sync header to know whether packet received is data (TLP or DLLP) or order set.
  \item All order sets bypass Descrambler except for symbols from 1 to 15 for TS ordered sets in GEN 3\&4\&5.
  \item The LMC un-stripes the data.
  \item The OS Decoder will decode the TS1OS \& TS2OS and give its information to Rx LTSSM.
  \item (Upstream) After Rx LTSSM receives 2 TS2OS with same link and lane numbers the LTSSM goes to Configuration.Lanenum.Accept substate.
  \item (Downstream) After Rx LTSSM receives 2 TS1OS with same link and lane numbers the LTSSM goes to Configuration.Lanenum.Accept substate.
\end{enumerate}
\item Configuration.Lanenum.Accept
\newline\textbf{Transmitter Side}
\begin{enumerate}
  \item (Downstream) Tx LTSSM keeps the same lane numbers and link number as in previous states and pass it to OS generator to generate TS1 ordered set.
  \item (Upstream) Tx LTSSM will keep sending its lane numbers and same link number then pass it to OS generator to generate TS1 ordered set.
  \item Tx LTSSM will inform the MUX to pass the TS1OS.
  \item Tx LTSSM will inform LMC to distribute the TS1OS on all lanes simultaneously.
  \item The TS1OS will pass through Scrambler then out to the PHY through the PIPE Tx data.
\end{enumerate}
\textbf{Receiver Side}
\begin{enumerate}
  \item The PIPE Rx Data checks data valid is asserted or not.
  \item For GEN 3\&4\&5, PIPE Rx Data checks for the sync header to know whether packet received is data (TLP or DLLP) or order set.
  \item All order sets bypass Descrambler except for symbols from 1 to 15 for TS ordered sets in GEN 3\&4\&5.
  \item The LMC un-stripes the data.
  \item The OS Decoder will decode the TS1OS \& TS2OS and give its information to Rx LTSSM.
  \item (Upstream) After Rx LTSSM receives 2 TS2OS with same link and lane numbers the LTSSM goes to Configuration.Complete substate.
  \item (Downstream) After Rx LTSSM receives 2 TS1OS with same link and lane numbers the LTSSM goes to Configuration.Complete substate.
\end{enumerate}
\item Configuration.Complete
\newline\textbf{Transmitter Side}
\begin{enumerate}
  \item (Downstream \& Upstream) Tx LTSSM will send the negotiated lane numbers and link number as in previous states and pass it to OS generator to generate TS2 ordered set.
  \item Tx LTSSM will inform the MUX to pass the TS2OS.
  \item Tx LTSSM will inform LMC to distribute the TS2OS on all lanes simultaneously.
  \item The TS2OS will pass through Scrambler then out to the PHY through the PIPE Tx data.
\end{enumerate}
\textbf{Receiver Side}
\begin{enumerate}
  \item The PIPE Rx Data checks data valid is asserted or not.
  \item For GEN 3\&4\&5, PIPE Rx Data checks for the sync header to know whether packet received is data (TLP or DLLP) or order set.
  \item All order sets bypass Descrambler except for symbols from 1 to 15 for TS ordered sets in GEN 3\&4\&5.
  \item The LMC un-stripes the data.
  \item The OS Decoder will decode the TS2OS and give its information to Rx LTSSM.
  \item (Downstream \& Upstream) After Rx LTSSM receives 8 TS2OS and Tx LTSSM send 16 TS2OS with the negotiated link and lane numbers the LTSSM goes to Configuration.Idle substate.
\end{enumerate}
\item Configuration.Idle
\newline\textbf{Transmitter Side}
\begin{enumerate}
  \item (Downstream \& Upstream) Tx LTSSM will inform OS generator to send Idle symbols.
  \item Tx LTSSM will inform the MUX to pass the Idle symbol.
  \item Tx LTSSM will inform LMC to distribute the Idle symbol on all lanes simultaneously.
  \item The Idle symbol will pass through Scrambler then out to the PHY through the PIPE Tx data.
\end{enumerate}
\textbf{Receiver Side}
\begin{enumerate}
  \item The PIPE Rx Data checks data valid is asserted or not.
  \item For GEN 3\&4\&5, PIPE Rx Data checks for the sync header to know whether packet received is data (TLP or DLLP) or order set.
  \item All order sets bypass Descrambler except for symbols from 1 to 15 for TS ordered sets in GEN 3\&4\&5.
  \item The LMC un-stripes the data.
  \item The OS Decoder will decode the Idle symbol then inform the Rx LTSSM that Idle symbol is received.
  \item (Downstream \& Upstream) After Rx LTSSM receives 8 Idle symbols and Tx LTSSM send 16 Idle symbols then Linkup signal is asserted and training is successfully completed.
\end{enumerate}
\end{enumerate}

